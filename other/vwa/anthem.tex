% Developing a digital synthesizer in C++

\documentclass{report}

\usepackage{listings}

\usepackage{color}

\usepackage{url}

\begin{document}

  \definecolor{custommagenta}{RGB}{230,70,165}
  \definecolor{customgreen}{RGB}{30,210,100}
  \definecolor{customgray}{gray}{0.5}

  \lstset{
  % Format settings
  language=C++,
  tabsize=4,
  breakatwhitespace=true,
  breaklines=true,
  frame=leftline,
  rulecolor=\color{green},
  captionpos=b,
  keepspaces=true,
  showstringspaces=false,
  numbers=left,
  numberstyle=\small,
  % Style settings
  basicstyle=\ttfamily,
  keywordstyle=\color{custommagenta}\ttfamily,
  stringstyle=\color{customgreen}\ttfamily,
  commentstyle=\color{customgray}\ttfamily,
  morecomment=[l][\color{red}]{\#}
  }

  % Macro for source code inclusion. Moves to main
  % directory then takes the further path as argument
  \newcommand{\code}[1]{\lstinputlisting{../../#1}}

  \title{Developing a digital synthesizer in C++}

  \author{
    Peter Goldsborough\\
    \texttt{petergoldsborough@hotmail.com}
  }

  \date{\today}

  \maketitle

  \chapter{Modulating sound}

  A variable: $A$

    \section{Envelopes}

    $$ \sqrt{4} = 2 $$

      \subsection{Single segments}

      Creating a digital synthesizer is simple \cite{bsynth}.

        \subsubsection{The EnvSeg class}
        \pagebreak
        \code{src/effects/reverb.cpp}

  \begin{thebibliography}{99}

    \bibitem{bsynth}

      Daniel R. Mitchell,

      \emph{BasicSynth: Creating a Music Synthesizer in Software}.

      Publisher: Author.

      1st Edition,

      2008.

    \bibitem{dspguide}

      Steven W. Smith

      \emph{The Scientist and Engineer's Guide to Digital Signal Processing}.

      California Technical Publishing,

      San Diego, California,

      2nd Edition,

      1999.

    \bibitem{sosfm}

      Gordon Reid.

      \emph{Synth Secrets, Part 12: An Introduction To Frequency Modulation}.

      \url{http://www.soundonsound.com/sos/apr00/articles/synthsecrets.htm}

      Accessed: 8 October 2014.

  \end{thebibliography}

\end{document}
